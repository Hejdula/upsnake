\documentclass[english, kiv, sem, he, iso690alph, pdf, viewonly]{fasthesis}
\title{Multiplayer Snake Game}
\author{Jan}{Hejdušek}
\supervisor{}
% \assignment{sw2025-02.pdf}

\usepackage{csquotes}
\usepackage{pdfpages}
\usepackage{listings}
\usepackage{syntax}

\nobastardtitle
\nocopyrightnotice

\newif\iffullbuild
\fullbuildtrue

\lstset{style=FASThesisLstStyle, numberblanklines=false, tabsize=4,
keywordstyle=\color{blue}}

\begin{document}
\iffullbuild
\frontpages[notm]
\tableofcontents
\fi

\chapter{Introduction}
\section{Overview}
This document is a documentation to a seminar work for the course
\lstinline|KIV/UPS|. It describes the implementation of a multiplayer
snake game.
The application operates on a client--server architecture, where a
central server manages the game state, rooms, and synchronization,
while clients provide a graphical interface for players.
The goal of this project is to demonstrate the understanding of
network programming and ability to create robust custom application
protocols.
\\\\
The document covers the following topics:

\begin{itemize}
  \item High--level architecture and technology stack.
  \item Detailed description of the communication protocol.
  \item Internal implementation details of both server and client.
  \item Short user guide for compilation and running the programs.
\end{itemize}

\section{Game Rules}
The game follows the classic Snake mechanics adapted for a
multiplayer environment.
Players control snakes on a shared grid, aiming to survive the longest.

Starting positions of the snakes and the apple are randomized.
All Snakes move at the same time in one of four directions. Eating
the apple increases the snake's length.
A player is eliminated upon collision with a wall, another snake, or themselves.
The last surviving player wins the round. If all remaining players
collide simultaneously, the game ends in a draw.

\chapter{Implementation}

The project is split into two distinct parts: a C++ server and a
Python client. They communicate over TCP/IP using a custom text--based
protocol. Both the client and the server closes the connection if the
protocol is violated or if the other side is inactive for too long.

\section{Server Architecture}

\subsection{Modularization}
The server codebase is organized into following files/classes:
\begin{itemize}
  \item \texttt{Server}: The core class that manages the life--cycle
    of the application, networking setup, and the main event loop. It
    acts as the central controller handling I/O events.
  \item \texttt{Game}: Encapsulates the logic for a single game room.
    It maintains the game state (grid, snake positions, apple) and
    processes game rules independently of the networking layer.
  \item \texttt{Connection}: Wraps a client's file descriptor and
    associated buffers, linking the socket to a
    \texttt{Player} instance.
  \item \texttt{Player}: Represents a user in the game, storing
    attributes like nickname, current direction, and the snake's body segments.
\end{itemize}

\subsection{Event Loop and Epoll}
To handle incoming connections and concurrent events the server uses
the Linux \texttt{epoll} API. The main
server loop waits for events on a single file descriptor set, which includes:
\begin{itemize}
  \item The listening socket (for new connections).
  \item Connected client sockets (for incoming messages).
  \item \texttt{timerfd} file descriptors (for game ticks, ping and timeouts).
\end{itemize}
This pooling allows the server to handle multiple game rooms
and clients within a single thread.

\subsection{Game Management}
The server manages multiple \lstinline|Game| instances (rooms). Each room
handles it own game state. The state is updated on timer events using
\lstinline|timerfd| file descriptor.

To ensure fairness and synchronization over
the network, the game tick processing is done purely on server.
Before advancing the game state, the server waits for clients to
confirm they recieved data about last game tick (up to a timeout),
also notifying other players for whom are they waiting.

\section{Client Architecture}
The client is implemented in Python using the \textbf{PyQt6} framework.
It relies on the Qt signal mechanism to synchronize state
between the networking logic and the user interface.

\subsection{NetworkWorker}
The \texttt{NetworkWorker} class runs in a background thread to
prevent blocking the main GUI event loop.
It manages the blocking TCP socket operations.

Incoming raw data is buffered and split by the pipe character
\lstinline+|+. Complete, valid messages (the first four characters
are checked if they match the protocol) are sent to the main thread with the
\texttt{msg\_received(msg)} signal.

The network thread also manages connection state, among other signals
emitting signal
\\ \lstinline|connection_unstable| to notify user about network
issues and signal \texttt{reconnect} to begin trying to reconnect to
the server. When reconnecting, the network thread is terminated and
reinitialized on connection attempts by the main thread.

\subsection{User Interface Structure}
The application uses a \texttt{QStackedWidget} to manage view transitions.
The primary widgets include:
\begin{itemize}
  \item \textbf{LoginWidget}: The entry screen for inputting nickname
    and server details.
  \item \textbf{RoomListWidget}: Displays the list of available rooms
    and allows the user to join one.
  \item \textbf{GameWidget}: The main container for the active
    session. It is composed of:
    \begin{itemize}
      \item \textbf{LobbyWidget}: Displays the player list within
        the room and buttons to start or leave the game.
      \item \textbf{GameBoard}: A custom widget that handles the
        rendering of snakes, apples, and grid lines using
        \texttt{QPainter}. Capable of handling keyboard events for movement.
    \end{itemize}
  \item \textbf{ReconnectWidget}: An overlay displayed during
    connection loss to offer manual cancellation and status updates.
\end{itemize}

\subsection{Application Flow}
The user experience follows a state-machine-like flow:
\begin{enumerate}
  \item \textbf{Connection}: The user initiates a connection from the
    Login screen.
  \item \textbf{Room Selection}: Upon handshake and setting a nick,
    the user browses available rooms.
  \item \textbf{Lobby}: Joining a room transitions to the Game view,
    initially in a "Gmae lobby" state where players wait for others.
  \item \textbf{Gameplay}: Once the game starts, the
    \texttt{GameBoard} receives focus. The state is updated purely
    via server \texttt{TICK} messages.
  \item \textbf{Game end}: After a match ends (Win/Draw), players
    remain in the room and can start a new game.
\end{enumerate}

\chapter{Communication Protocol}
The communication uses a custom text-based protocol over TCP.
All messages start with a 4 characters determining type of the
message. Whole messages are delimited by a pipe character
'\texttt{|}' and message fields are separated with a white space character.

\section{Client Requests}
Messages sent from the Client to the Server.

\begin{description}
  \item\texttt{NICK <nickname>} \\
    Registers the client on the server. 
    Based on the answer from server, the client infers its state. Server responds
    with \texttt{ROOM} or \texttt{LOBY} message, additionally
    \texttt{TICK} message if the game in the room is active.

  \item\texttt{LIST} \\
    Request a list of available rooms. Server responds with a \texttt{ROOM} message
    containing sizes of all rooms.

  \item\texttt{JOIN <room\_id>} \\
    Request to join a specific room. Server responds with a \texttt{LOBY} message
    listing players in the room if successful, or \texttt{FULL} if the room is full.

  \item\texttt{LEAV} \\
    Leave the current room. Server confirms with a \texttt{LEFT} message.

  \item\texttt{MOVE <direction>} \\
    Sends inputs (\texttt{U}, \texttt{D}, \texttt{L}, \texttt{R}). Server
    acknowledges with a \texttt{MOVD} message.

  \item\texttt{STRT} \\
    Request to start the game. Server responds with \texttt{STRT OK} on success
    or \texttt{STRT FAIL} on failure.

  \item\texttt{QUIT} \\
    Gracefully disconnects.
\end{description}

\section{Server Notifications}
Messages sent from the Server to client.

\begin{description}
  \item\texttt{TICK <ax> <ay> [<nick> <hx> <hy> <status><body>] ...} \\
    Game state update. In active game, each game tick server sends game state to all players in room. 
    Client must acknowledge with \texttt{TACK}, if server does not recieve the acknowledge from all players before next game tick, it notifies all other players and waits.

  \item\texttt{PING} \\
    Connection liveliness check. Client replies with \texttt{PONG}.

  \item\texttt{WAIT <nick> ...} \\
    Server waiting for lagging players. Client acknowledges with \texttt{ZZZZ}.

  \item\texttt{WINS <nick>} \\
    Game ended, with winning player nick. Client acknowledges the end of game with \texttt{SSSS}.

  \item\texttt{DRAW} \\
    Game ended in a draw. Client acknowledges the end of game with \texttt{SSSS}.
\end{description}

\section{Game State Encoding}
The \texttt{TICK} message carries the full compressed game state:
\begin{lstlisting}
TICK <ax> <ay> [<nick> <hx> <hy> <status><body>] ...
\end{lstlisting}
Where:
\begin{itemize}
  \item \texttt{ax, ay}: Apple coordinates.
  \item \texttt{nick}: Player nickname.
  \item \texttt{hx, hy}: Snake head coordinates.
  \item \texttt{status}: \texttt{H} (Alive) or \texttt{E} (Eliminated).
  \item \texttt{body}: String of directions (U, D, L, R) representing
    body segments.
\end{itemize}

\section{Formal Grammar (BNF)}
Formal definition of the protocol using Backus-Naur Form
(BNF) notation.

\setlength{\grammarparsep}{5pt plus 1pt minus 1pt}
\setlength{\grammarindent}{12em}

\begin{grammar}
  <message>       ::= <content> `|'

  <content>       ::= <client-msg>
  \alt <server-msg>

  <client-msg>    ::= `NICK' <sp> <nick>
  \alt `LIST'
  \alt `JOIN' <sp> <int>
  \alt `LEAV'
  \alt `MOVE' <sp> <dir>
  \alt `STRT'
  \alt `QUIT'
  \alt `TACK'
  \alt `PONG'
  \alt `ZZZZ'
  \alt `SSSS'

  <server-msg>    ::= `ROOM' \{ <sp> <int> \}
  \alt `LOBY' \{ <sp> <nick> \}
  \alt `WAIT' \{ <sp> <nick> \}
  \alt `PING'
  \alt `WINS' <sp> <nick>
  \alt `DRAW'
  \alt `TICK' <sp> <int> <sp> <int> \{ <p-state> \}
  \alt `FULL'
  \alt `LEFT'
  \alt `MOVD'
  \alt `STRT' <sp> (`OK' | `FAIL')

  <p-state>       ::= <sp> <nick> <sp> <int> <sp> <int> <sp> <stat> <dirs>

  <stat>          ::= `H' | `E'

  <dirs>          ::= \{ <dir> \}

  <dir>           ::= `U' | `D' | `L' | `R'

  <sp>            ::= ` '
\end{grammar}

\section{Reconnection Handling}
The imlementation with the protocol support session recovery for clients that experience
temporary network issues.

The server maintains information about players who unexpectedly
disconnect for a period of time. During this window, the
\texttt{Player} object and its state are preserved in memory.

If a client connects and sends a \texttt{NICK} that matches a
preserved or an existing player:
\begin{enumerate}
  \item The old socket connection (if some) is forced closed.
  \item The existing \texttt{Player} instance is bound to the new connection.
  \item The current context state (\texttt{ROOM} list, \texttt{LOBY}
    content or current game \texttt{TICK})
    is sent to the reconnected client.
\end{enumerate}
This allows a user to restart their client or recover from a
connection drop and immediately
resume their position in a running game or lobby.

The client checks how much time passed since the last message from server. 
If greater than given timeout, it closes the connection and tries to open a new one multiple times with the server. 

\chapter{User guide}
\section{Compilation/installation}

\subsection{Server}
The server is written in C++17 and requires a GCC compiler.
\begin{console}{Build the server}
  `\uxprompt`cd server
  `\uxprompt`make
\end{console}

\subsection{Client}
The client requires Python3 interpret and PyQt6 library.
\begin{console}{Install dependencies}
  `\uxprompt`cd client
  `\uxprompt`pip install -r requirements.txt
\end{console}

\section{Usage}

\subsection{Running the Server}
Start the server providing port and IP address (optional).
\begin{console}{Start Server}
  `\uxprompt`./server/server 8888 127.0.0.1
  Listening on: 127.0.0.1:8888
\end{console}

\subsection{Running the Client}
Start the client application:
\begin{console}{Run Client}
  `\uxprompt`python client/client_app.py
\end{console}
A GUI window will appear. Enter a nickname, the server address and
port to connect.

\chapter{Conclusion}
The project fulfills all requirements specified by the assignment.

I am satisfied with the implementation of concurrency on both the
server and the client.
On the server, using \texttt{epoll} and timer file descriptors
simplified the design significantly.
This event--driven approach removed the need for complex threading,
as the server only wakes up for messages or events.
On the client, the separate networking thread worked well with PyQt signals.

With the current implementation on the server, it would not scale
well with increasing number of players as many lookups are
brute--force as optimization was not a concern.

The protocol works reliably, but its design is not optimal.
Using a text delimiter to separate messages and spaces for fields is
inefficient.
A length prefix format would be more robust for determining message
boundaries and finding out invalid or corrupted messages before
reaching the delimiter.

\end{document}
